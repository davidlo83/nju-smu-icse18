To evaluate the effectiveness of \TRANPCNN, we conduct experiments on open source software projects and compare it with several state-of-the-art bug localization methods.

\subsection{Research Questions}
%\dl{Ferdian, please add details on why the RQs are interesting and the purpose of the RQs}

Our experiments are designed to address the following research questions:

\vspace{0.2cm}\noindent\textit{\textbf{RQ1}: Is there a need for a specialized technique for cross-project bug localization?}

If a model learned from one project can be used for others project, then there is no need for a specialized technique for cross-project bug localization. Thus, before we consider other research questions, we validate the need for our proposed approach by empirically evaluating the effectiveness of a model learned from one project to localize bugs in other projects.

\vspace{0.2cm}\noindent\textit{\textbf{RQ2}: Does the cross-project feature fusion layer improve the bug localization performance?}

\dl{Xuan, please help to motivate this research question. I can't motivate it since I believe the details of the approach is going to change.}

In Section 3, we propose to employ heterogeneous predicting layers adaptation that apply two fully-connected networks for prediction from source and target projects separately, which is the key part of TRANP-CNN. In this research question, we evaluate whether the heterogeneous predicting adaptation layers help improve the bug localization performance by comparing the results previous NPCNN and SimpleTrans model. 

\vspace{0.2cm}\noindent\textit{\textbf{RQ3}: Can TRANP-CNN outperform other bug localization methods?}

A number of bug localization methods have been proposed in the literature. In this research question, we evaluate whether and to what extent can our proposed approach TRANP-CNN outperforms the state-of-the-art methods designed for bug localization and those that can be adapted for bug localization. 

\subsection{Datasets}
We consider datasets that were previously studied and manually vetted by Herzig et al.~\cite{HerzigJZ13} and Kochhar et al.~\cite{KochharTL14}. Herzig et al. highlighted that many reports in issue tracking systems are wrongly labelled as bugs when they are actually feature requests (and vice versa). Kochhar et al. highlighted that many bug reports are already {\em fully localized}, i.e., developers have already identified all buggy source code files in the bug reports. For such bug reports, bug localization tool is no longer needed.

To deal with these biases, Herzig et al. have manually classified 5,591 issue reports from JIRA issue tracking systems of three projects (HTTPClient, Jackrabbit, and Lucene-Java) and 1,810 issue reports from Bugzilla issue tracking systems of two projects (Rhino and Tomcat 5) and released the dataset publicly\footnote{\url{https://www.st.cs.uni-saarland.de/softevo/bugclassify}}. Kochhar et al. have analyzed 1,191 reports from JIRA issue tracking systems that were confirmed by Herzig et al. as bug reports. They focused on reports from JIRA issue tracking systems since a number of studies have shown that reports in Bugzilla are often poorly linked with commits that fix them~\cite{BachmannBRDB10,BirdBADBFD09}, while bug reports in JIRA are often more well-linked as JIRA provides a utility to better connect bug reports to their corresponding commits~\cite{BissyandeTWLJR13}. Kochhar et al. have identified a set of 398 bug reports that are already {\em fully localized} and another set of 793 bug reports that are either {\em partially localized} (i.e., reports where some of the files containing the bugs are explicitly mentioned in the report) or {\em not localized} (i.e., reports which do not explicitly specify any of the buggy files.). They also have released this dataset publicly\footnote{\url{https://github.com/smusis/buglocalizationbiases}}.

In this work, we consider the 793 bug reports from three software systems: 63 from HTTPClient (H), 534 from Jackrabbit (J), and 196 from Lucene-Java (L), which are provided by Kochhar et al. HTTPClient\footnote{\url{http://hc.apache.org/httpcomponents-client-ga/index.html}} is a library for implementing the client side of HTTP standard, while Jackrabbit\footnote{\url{https://jackrabbit.apache.org/jcr/index.html}} is a content repository, and Lucene is a text search engine\footnote{\url{ http://lucene.apache.org/}}. The details of the reports considered in this study are shown in Table~\ref{tab:reports}. Although the number of reports considered is fewer than those considered in several prior work, these reports are of high-quality. They has been manually vetted before and are absent from well-known biases identified by Herzig et al. and Kochhar et al. Most past studies have ignored these well-known biases and thus introduce a threat to the validity of their findings.

%that have been fully localized by developers. They focus on bug reports
%containing a total of 5,591 reports from JIRA issue tracking systems of
%We only consider projects with JIRA issue tracking systems since links between bug reports and their bug fixing commits stored in them are typically more reliable than those stored in Bugzilla issue tracking systems -- c.f.,~\cite{BissyandeTWLJR13}. The details of the reports considered in this study are shown in Table~\ref{tab:reports} and they have been used before by Kochhar et al.~\cite{KochharTL14}.

%\begin{table}
%\caption{Bug Report Dataset}\label{tab:reports}
%\begin{tabular}{|l|l|l}
%\hline
%{\bf Project} & {\bf \# Reports} \\
%\hline HTTPClient & 63\\
%\hline Jackrabbit & 534\\
%\hline Lucene-Java & 196\\
%\hline
%\end{tabular}
%\end{table}

%\dl{Xuan, I thought we are also using your previous datasets? If we only use Pavneet's dataset, do we use all of them or only some of them that are not biased? Pavneet showed that some of the dataset is biased ... reviewers may be concerned with this bias ... Bias 2 mentioned in this paper (\url{http://ink.library.smu.edu.sg/cgi/viewcontent.cgi?article=3425&context=sis_research}) is particularly important.}

%\xh{Currently we only use Paveneet's datasets (H,J and L), and yes we have considered the bias, and we only use the unbiased data sets. The ``fully localized'' bug reports are filtered. Our previous data sets are bias, so we are not sure if we use here is suitable.  If necessary, we can conduct several comparison experiments ono more data sets. }

%\dl{Ferdian, please add details datasets. Please see the following papers for details of the datasets:~\cite{huo2016learning,KochharTL14}.}



\subsection{Evaluation Metrics}

The evaluation metrics are presented here.

\dl{Ferdian, please add details on evaluation metrics.}



\subsection{Baselines}

We compare our proposed model \TRANPCNN with the following baseline methods:
\begin{itemize}
  \item VSM (Vector Space Model)~\cite{rao2011retrieval}: a baseline method that first uses a Vector Space Model to represent textual contents of bug reports and source code, and then employs Logistic Regression to predict buggy source code when given a new bug report.
  \item Burak (Burak Filter)~\cite{peters2013better}: a state-of-the-art method for cross-project and cross-company defect prediction problem. It filters training sets using Burak filter by employing $k$-nearest neighbour algorithm to select instances in the source project that are most similar to instances in the test project.
  \item TCA+ (Transfer Component Analysis)~\cite{NamPK13}: a state-of-the-art transfer learning method in software engineering, which first normalizes the training sets and employs TCA to map source and target project into a single feature space and then applies Logistic Regression for bug localization (same settings suggested in their paper). 
  \item TCA+$^P$ (Transfer Component Analysis with Multi-Layer Perceptron (MLP)): a state-of-the-art transfer learning method in software engineering, which first normalizes training sets and employs TCA to map source and target project into a single feature space and then applies MLPs for bug localization~(same settings with fully-connected layers in \TRANPCNN).
   \item TCA+$^D$ (Transfer Component Analysis with Deep features): a state-of-the-art transfer learning method in software engineering, which first normalizes training sets and employs TCA to map source and target project into a single feature space and then applies Logistic Regression for bug localization (using deep features extracted from CNN instead of VSM features).
  %\item NP-CNN (Natural and Programming language Convolutional Neural Network)~\cite{huo2016learning}: a state-of-the-art deep model for bug localization, which uses a source project for training a deep CNN to localize buggy source code in the target project.
\end{itemize}
%\ft{Settings would probably better be grouped in experimental settings section. There is no citation for TCA. }

\subsection{Experimental Settings}
For the parameters of baseline methods (VSM, Burak, TCA+, TCA+$^{(P)}$, TCA+$^{(D)}$), we use the same parameter settings suggested in their work~\cite{rao2011retrieval,NamPK13}. For NP-CNN, we also use the parameter settings suggested in its paper~\cite{huo2016learning}.

For the \TRANPCNN model, we employ the most commonly used ReLU $\sigma(x)=\max(0,x)$ as the activation function and the filter windows size $d$ is set to 3, 4, 5, with 100 feature maps each in within-project feature extraction layers. The number of neurons in fully-connected network is set the same as the number of neurons in CNN. In addition, a drop-out method is also applied to prevent co-adaption of hidden units by randomly dropping out values in fully-connected layers. The drop-out probability $p$ is set to 0.25.

For data partition, we use all data from a source project and 20\% data from a target project as training sets, and locate buggy codes in 80\% remaining data from the target project. This process is repeated five times to reduce the effect of randomness. We then report the average results for comparison. 

\section{Experimental Results}

\subsection{Experimental Results for Research Questions}

\begin{table}[htbp]
  \centering
  \caption{Performance Comparisons between within-project and cross-project bug localization.}
  \resizebox{!}{0.5\columnwidth}{
    \begin{tabular}{c|l|c|c|c|c|c}
    \toprule
    Tasks & \textit{Methods} & \multicolumn{1}{l}{\textit{Top 1}} & \multicolumn{1}{l}{\textit{Top 5}} & \multicolumn{1}{l}{\textit{Top 10}} & \multicolumn{1}{l}{\textit{MAP}} & \multicolumn{1}{l}{\textit{MRR}} \\
    \midrule
    \multirow{3}[0]{*}{\textbf{J}$\rightarrow$\textbf{H}} & NP-CNN & 0.317  & 0.362  & 0.508  & 0.276  & 0.352  \\
          & NP-CNN$^{partial}$ & 0.204  & 0.258  & 0.313  & 0.202  & 0.292  \\
          & NP-CNN$^{full}$ & \textbf{0.533}  & \textbf{0.617}  & \textbf{0.650}  & \textbf{0.472}  & \textbf{0.580}  \\
          \midrule
    \multirow{3}[0]{*}{\textbf{L}$\rightarrow$\textbf{H}} & NP-CNN & 0.142  & 0.192  & 0.345  & 0.161  & 0.218  \\
          & NP-CNN$^{partial}$ & 0.204  & 0.258  & 0.313  & 0.202  & 0.292  \\
          & NP-CNN$^{full}$ & \textbf{0.533}  & \textbf{0.617}  & \textbf{0.650}  & \textbf{0.472}  & \textbf{0.580}  \\
          \midrule
    \multirow{3}[0]{*}{\textbf{H}$\rightarrow$\textbf{J}} & NP-CNN & 0.167  & 0.287  & 0.349  & 0.247  & 0.277  \\
          & NP-CNN$^{partial}$ & 0.035  & 0.211  & 0.302  & 0.155  & 0.189  \\
          & NP-CNN$^{full}$ & \textbf{0.508}  & \textbf{0.587}  & \textbf{0.679}  & \textbf{0.462}  & \textbf{0.557}  \\
          \midrule
    \multirow{3}[0]{*}{\textbf{L}$\rightarrow$\textbf{J}} & NP-CNN & 0.152  & 0.182  & 0.318  & 0.176  & 0.221  \\
          & NP-CNN$^{partial}$ & 0.035  & 0.211  & 0.302  & 0.155  & 0.189  \\
          & NP-CNN$^{full}$ & \textbf{0.508}  & \textbf{0.587}  & \textbf{0.679}  & \textbf{0.462}  & \textbf{0.557}  \\
          \midrule
    \multirow{3}[0]{*}{\textbf{H}$\rightarrow$\textbf{L}} & NP-CNN & 0.173  & 0.246  & 0.390  & 0.196  & 0.329  \\
          & NP-CNN$^{partial}$ & 0.097  & 0.219  & 0.335  & 0.095  & 0.109  \\
          & NP-CNN$^{full}$ & \textbf{0.289}  & \textbf{0.484}  & 0.611  & \textbf{0.287}  & \textbf{0.387}  \\
          \midrule
    \multirow{3}[0]{*}{\textbf{J}$\rightarrow$\textbf{L}} & NP-CNN & 0.110  & 0.255  & 0.323  & 0.141  & 0.176  \\
          & NP-CNN$^{partial}$ & 0.097  & 0.219  & 0.335  & 0.095  & 0.109  \\
          & NP-CNN$^{full}$ & \textbf{0.289}  & \textbf{0.484}  & \textbf{0.611}  & \textbf{0.287}  & \textbf{0.387}  \\
          \bottomrule
    \end{tabular}%
    }

  \label{tab:results1}%
\end{table}%


\begin{figure}[hbt]
\centering
\includegraphics[width = 0.9\columnwidth]{pic/results1_avg.pdf}
\caption{Performance comparisons between within-project and cross-project bug localization.}
\label{fig:results1}
\end{figure}


\textbf{RQ1}: \textit{Is there a need for cross-project bug localization?}

To answer this research question, we compare the results of using NP-CNN for bug localization in different settings.

\begin{itemize}
  \item NP-CNN: Directly employs NP-CNN for cross-project bug localization, which means directly training the model on the data from a source project and locating the bugs in a target project.
  \item NP-CNN$^{partial}$: Employs NP-CNN using partial data of a target project, which means training based on a partial data (20\%) from the target project, and localizing buggy files without using data from a source project.
  \item NP-CNN$^{full}$: Employs NP-CNN using full data of a target project. In this setting, we conduct a 5-fold cross-validation for comparison.
\end{itemize}

The results are detailed in Table~\ref{tab:results1} and Figure~\ref{fig:results1}. There are six tasks in the table.%, in which $\textbf{H}$ represents project \textit{HTTPClient}, $\textbf{J}$ represents project \textit{Jackrabbit} and $\textbf{L}$ represents \textit{Lucene-Java}. 
The task $\textbf{H} \rightarrow \textbf{J}$ represents using \textit{HTTPClient} as source project and predicts the location of buggy files in \textit{Jackrabbit}. The results show that the performance of bug localization using full data of the target project is the best and has a large gap against the performance of using only a partial data. For cross-project bug localization, the performance of NP-CNN that directly uses a source project is better than NP-CNN$^{partial}$, showing that cross-project data is useful to improve bug localization performance. However, directly using within-project bug localization does not work as well as NP-CNN$^{full}$. These results suggest that there is a need for cross-project bug localization, since directly using within-project bug localization method does not show a good performance.

\textbf{RQ2}: \textit{Do the project-specific prediction layers improve the bug localization performance?}

To answer this research question, we compare the results of \TRANPCNN with NP-CNN. The structural difference between \TRANPCNN and NP-CNN is on the two fully-connected networks in \TRANPCNN that combine deep features from source and target project (i.e., project-specific prediction layer), which counter the influences of cross-project data that may have different distribution and leads to a bias in performance. The results are detailed in Table~\ref{tab:results2}.

%\ft{It would be good if we can explain why such structure can counter bias. Perhaps we should explain this first in approach (theory) and then highlight it again accompanied with experimental result (empirical).}

\begin{table}[htbp]
  \centering
  \caption{Performance comparisons with previous deep models.}
  \resizebox{!}{0.35\columnwidth}{
    \begin{tabular}{c|l|c|c|c|c|c}
    \toprule
    Tasks & \textit{Methods} & \textit{Top 1} & \textit{Top 5} & \textit{Top 10} & \textit{MAP} & \textit{MRR} \\
    \midrule
    \multirow{2}[0]{*}{\textbf{J}$\rightarrow$\textbf{H}} & NP-CNN & 0.317  & 0.362  & 0.508  & 0.276  & 0.352  \\
%          & SimpleTrans & 0.354 & 0.396 & 0.563 & 0.298 & 0.395 \\
          & \TRANPCNN & \textbf{0.500}   & \textbf{0.583} & \textbf{0.625} & \textbf{0.376} & \textbf{0.543} \\
          \midrule
    \multirow{2}[0]{*}{\textbf{L}$\rightarrow$\textbf{H}} & NP-CNN & 0.142  & 0.192  & 0.345  & 0.161  & 0.218  \\
%          & SimpleTrans & 0.163 & 0.146 & 0.354 & 0.141 & 0.246 \\
          & \TRANPCNN & \textbf{0.275} & \textbf{0.35}  & \textbf{0.488} & \textbf{0.242} & \textbf{0.332} \\
          \midrule
    \multirow{2}[0]{*}{\textbf{H}$\rightarrow$\textbf{J}} & NP-CNN & 0.167  & 0.287  & 0.349  & 0.247  & 0.277  \\
%          & SimpleTrans & 0.133 & 0.324 & 0.365 & 0.273 & 0.301 \\
          & \TRANPCNN & \textbf{0.396} & \textbf{0.443} & \textbf{0.514} & \textbf{0.371} & \textbf{0.434} \\
          \midrule
    \multirow{2}[0]{*}{\textbf{L}$\rightarrow$\textbf{J}} & NP-CNN & 0.152  & 0.182  & 0.318  & 0.176  & 0.221  \\
%          & SimpleTrans & 0.144 & 0.204 & 0.382 & 0.247 & 0.249 \\
          & \TRANPCNN & \textbf{0.460}  & \textbf{0.462} & \textbf{0.488} & \textbf{0.404} & \textbf{0.478} \\
          \midrule
    \multirow{2}[0]{*}{\textbf{H}$\rightarrow$\textbf{L}} & NP-CNN & 0.173  & 0.246  & 0.390  & 0.196  & 0.329  \\
%          & SimpleTrans & 0.197 & 0.323 & 0.426 & 0.152 & 0.313 \\
          & \TRANPCNN & \textbf{0.361} & \textbf{0.445} & \textbf{0.535} & \textbf{0.279} & \textbf{0.414} \\
          \midrule
    \multirow{2}[0]{*}{\textbf{J}$\rightarrow$\textbf{L}} & NP-CNN & 0.110  & 0.255  & 0.323  & 0.141  & 0.176  \\
%          & SimpleTrans & 0.140  & 0.282 & 0.342 & 0.163 & 0.224 \\
          & \TRANPCNN & \textbf{0.301} & \textbf{0.410}  & \textbf{0.517} & \textbf{0.247} & \textbf{0.368} \\
          \bottomrule
    \end{tabular}%
    }
  \label{tab:results2}%
\end{table}%

\begin{figure}[hbt]
\centering
\includegraphics[width = 0.9\columnwidth]{pic/results2_avg.pdf}
\caption{Performance comparisons with deep models.}
\label{fig:results2}
\end{figure}

The results show that \TRANPCNN performs better than NP-CNN, which suggest that the project-specific prediction layer can improve the performance of cross-project bug localization. This layer employs two fully-connected networks to learn a separate prediction function for source and target project. This structure helps to combat bias in data distribution, as evidenced by the higher performance of \TRANPCNN as compared to NP-CNN.  

\textbf{RQ3}: \textit{Can \TRANPCNN outperform other bug localization methods?}

To answer this research question, we compare the results of \TRANPCNN with state-of-the-art methods: VSM, Burak, TCA+, TCA+$^P$, and TCA+$^D$. %VSM is a baseline technique used in a within-project bug localization and we employ it on cross-project bug localization. Burak and TCA+ have been shown to have a good performance on cross-project and cross-company defect prediction. We adopt them for cross-project bug localization. For fair comparison with TCA+, we use same normalization strategy and classifier in their original paper (Logistic Regression) and MLP (same as \TRANPCNN). 
The results are detailed in Table~\ref{tab:results3}. 

%\ft{We should directly use abbreviations of baseline names since they have been introduced in Baselines section. Explanations about baselines are also redundant. }

According to the results, we have several findings: 1.) Burak and TCA techniques perform better than VSM, indicating that transfer learning algorithms can improve the performance of cross-project bug localization; 2.) \TRANPCNN outperforms TCA+$^P$, which shows that the high-level features extracted from CNN are more informative, providing a better representation that leads to better bug localization performance; 3.) TCA+$^D$ uses deep features extracted from CNN and the performance is not as well as \TRANPCNN, which further proves that the project-specific prediction layer improves bug localization performance; 4.) \TRANPCNN obtains the best average values in terms of all evaluation metrics. Comparing to the best baseline TCA+$^{D}$, \TRANPCNN improves the results by 24.6\%, 22.6\%, 20.9\%, 21.9\%, and 17.2\% in terms of Top-1, Top-2, Top-5, MAP, and MRR, respectively. According to Mann-Whitney U-test, we find that \TRANPCNN is significantly better in terms of all evaluation metrics. The results suggest that \TRANPCNN outperforms other traditional bug localization methods and transfer learning techniques in software engineering.

%\ft{We can highlight the amount of improvement that \TRANPCNN achieves as compared to the best baseline.}

\begin{table}[htbp]
  \centering
  \caption{Performance comparisons with traditional bug localization models.}
  \resizebox{!}{0.9\columnwidth}{
    \begin{tabular}{c|l|c|c|c|c|c}
    \toprule
    Tasks & \textit{Methods} & \multicolumn{1}{c|}{\textit{Top 1}} & \multicolumn{1}{c|}{\textit{Top 5}} & \multicolumn{1}{c|}{\textit{Top 10}} & \multicolumn{1}{c|}{\textit{MAP}} & \multicolumn{1}{c}{\textit{MRR}} \\
    \midrule
    \multirow{6}[0]{*}{\textbf{J}$\rightarrow$\textbf{H}} & VSM   & 0.098  & 0.157  & 0.177  & 0.087  & 0.143  \\
          & Burak & 0.110  & 0.126  & 0.138  & 0.116  & 0.121  \\
          & TCA+ & 0.120  & 0.212  & 0.144  & 0.157  & 0.162  \\
          & TCA+$^P$ & 0.114  & 0.133  & 0.154  & 0.123  & 0.176  \\
          & TCA+$^D$ & 0.122  & 0.225  & 0.271  & 0.168  & 0.248  \\
          & \TRANPCNN & \textbf{0.500}  & \textbf{0.583}  & \textbf{0.625}  & \textbf{0.376}  & \textbf{0.543}  \\
          \midrule
    \multirow{6}[0]{*}{\textbf{L}$\rightarrow$\textbf{H}} & VSM   & 0.059  & 0.098  & 0.237  & 0.099  & 0.112  \\
          & Burak & 0.113  & 0.203  & 0.242  & 0.143  & 0.143  \\
          & TCA+ & 0.120  & 0.188  & 0.244  & 0.151  & 0.158  \\
          & TCA+$^P$ & 0.128  & 0.200  & 0.252  & 0.161  & 0.167  \\
          & TCA+$^D$ & 0.102  & 0.237  & 0.367  & 0.161  & 0.202  \\
          & \TRANPCNN & \textbf{0.275}  & \textbf{0.350}  & \textbf{0.488}  & \textbf{0.242}  & \textbf{ 0.332}  \\
          \midrule
    \multirow{6}[0]{*}{\textbf{H}$\rightarrow$\textbf{J}} & VSM   & 0.035  & 0.211  & 0.232  & 0.165  & 0.129  \\
          & Burak & 0.130  & 0.150  & 0.206  & 0.225  & 0.195  \\
          & TCA+ & 0.115  & 0.162  & 0.209  & 0.239  & 0.244  \\
          & TCA+$^P$ & 0.114  & 0.154  & 0.203  & 0.237  & 0.241  \\
          & TCA+$^D$ & 0.111  & 0.135  & 0.157  & 0.168  & 0.185  \\
          & \TRANPCNN & \textbf{0.396}  & \textbf{0.443}  & \textbf{0.514}  & \textbf{0.371}  & \textbf{0.434}  \\
          \midrule
    \multirow{6}[0]{*}{\textbf{L}$\rightarrow$\textbf{J}} & VSM   & 0.197  & 0.212  & 0.293  & 0.167  & 0.216  \\
          & Burak & 0.161  & 0.132  & 0.368  & 0.170  & 0.187  \\
          & TCA+ & 0.136  & 0.183  & 0.370  & 0.170  & 0.179  \\
          & TCA+$^P$ & 0.114  & 0.116  & 0.397  & 0.138  & 0.191  \\
          & TCA+$^D$ & 0.178  & 0.236  & 0.469  & 0.227  & 0.256  \\
          & \TRANPCNN & \textbf{0.460}  & \textbf{0.462}  & \textbf{0.488}  & \textbf{0.404}  & \textbf{0.478}  \\
          \midrule
    \multirow{6}[0]{*}{\textbf{H}$\rightarrow$\textbf{L}} & VSM   & 0.083  & 0.278  & 0.393  & 0.154  & 0.136  \\
          & Burak & 0.105  & 0.226  & 0.272  & 0.123  & 0.222  \\
          & TCA+ & 0.136  & 0.208  & 0.383  & 0.170  & 0.279  \\
          & TCA+$^P$ & 0.143  & 0.226  & 0.394  & 0.171  & 0.288  \\
          & TCA+$^D$ & 0.162  & 0.207  & 0.345  & 0.229  & 0.292  \\
          & \TRANPCNN & \textbf{0.361}  & \textbf{0.445}  & \textbf{0.535}  & \textbf{0.279}  & \textbf{0.414}  \\
          \midrule
    \multirow{6}[0]{*}{\textbf{J}$\rightarrow$\textbf{L}} & VSM   & 0.038  & 0.077  & 0.154  & 0.124  & 0.204  \\
          & Burak & 0.138  & 0.161  & 0.176  & 0.168  & 0.226  \\
          & TCA+ & 0.135  & 0.111  & 0.172  & 0.169  & 0.222  \\
          & TCA+$^P$ & 0.136  & 0.132  & 0.192  & 0.173  & 0.237  \\
          & TCA+$^D$ & 0.142  & 0.297  & 0.308  & 0.238  & 0.293  \\
          & \TRANPCNN & \textbf{0.301}  & \textbf{0.410}  & \textbf{0.517}  & \textbf{0.247}  & \textbf{0.368}  \\

          \bottomrule
    \end{tabular}%
    }
  \label{tab:results3}%
\end{table}%

\begin{figure}[hbt]
\centering
\includegraphics[width = 0.9\columnwidth]{pic/results3_avg.pdf}
\caption{Performance comparisons with traditional bug localization models.}
\label{fig:results3}
\end{figure}

