\documentclass[sigconf]{acmart}

\usepackage{booktabs} % For formal tables
\usepackage{multirow}
\usepackage[american]{babel}
\usepackage{color}
\usepackage{balance}


\newcommand{\dl}[1]{\textcolor{blue}{{David says: {#1}}}}
\newcommand{\ml}[1]{\textcolor{red}{{Ming says: {#1}}}}
\newcommand{\xh}[1]{\textcolor{brown}{{Xuan says: {#1}}}}
\newcommand{\ft}[1]{\textcolor{green}{{Ferdian says: {#1}}}}
\newcommand{\TRANPCNN}{TRANP-CNN }

\settopmatter{printacmref=false} % Removes citation information below abstract
%\renewcommand\footnotetextcopyrightpermission[1]{} % removes footnote with conference information in first column
\pagestyle{plain} % removes running headers


\begin{document}
\title{Deep Transfer Bug Localization}

\begin{abstract}
Many projects often receive more bug reports than what they can handle. To help debug and close bug reports, a number of bug localization techniques have been proposed. These techniques analyze a bug report and return a ranked list of potentially buggy source code files. Recent development on bug localization has resulted in the construction of effective supervised approaches that use historical data of manually localized bugs to boost performance. Unfortunately, as highlighted by Zimmermann et al., sufficient bug data is often unavailable for many projects and companies. This raises the need for {\em cross-project bug localization} -- the use of data from a project to help locate bugs in another project. To fill this need, we propose a deep transfer learning approach for cross-project bug localization. Our proposed approach named \TRANPCNN\ extracts transferable semantic features from source project and fully exploits labeled data from target project for effective cross-project bug localization. We have evaluated \TRANPCNN\ on curated bug datasets from Herzig et al. and Kochhar et al. Our experiments show that averaging across the datasets, \TRANPCNN\ can locate buggy files correctly at top-1, top-5, and top-10 positions for 38.22\%, 44.88\%, 52.78\% of the bugs respectively. It can outperform a state-of-the-art bug localization solution based on deep learning and several other advanced alternative solutions by 61.13\% to 349.61\% considering various standard evaluation metrics.
\end{abstract}

\maketitle

\section{Introduction}\label{sec.intro}
%Problem with Bug, Need for bug localization
An active software project often receives numerous bug reports daily~\cite{AnvikHM05}. To resolve each report, developers often need to spend much time and effort~\cite{Tassey02}. One main task that developers need to do during debugging is to identify code that needs to be fixed to resolve the bug. This task, often referred to as {\em bug localization}, is a non-trivial one as relevant files need to be identified out of a collection of hundreds or even thousands of files.

%Existing work on bug localization, Recent trend on supervised bug localization (IJCAI16,17)
To help developers in locating bugs, various automated solutions have been proposed~\cite{JonesH05,lukins2008source,rao2011retrieval,SahaLKP14,huo2016learning}. Many of them analyze the description of a bug report to identify source code files relevant to it~\cite{lukins2008source,rao2011retrieval,SahaLKP14,huo2016learning}. These text-based solutions can be further divided into two families: unsupervised and supervised. Unsupervised solutions, which were historically proposed first, typically employ information retrieval techniques to identify files that contain many words that appear in the bug report~\cite{lukins2008source,rao2011retrieval,SahaLKP14}. More recently, supervised approaches are introduced~\cite{huo2016learning}. These approaches use a collection of bug reports and their relevant source code files as training data. This data is then used to learn a good model that can map new bug reports to their respective relevant source code files. Supervised approaches have been shown to be superior than unsupervised ones.

%Limitation with supervised - cold start problem, no data.
One limitation of a supervised approach is the need for sufficient and high quality training data. Insufficient or low quality data can be detrimental to its effectiveness. This problem is particularly important when a bug localization approach needs to be applied to new projects having a limited bug fixing history. Unfortunately, this issue, often referred to as the {\em cold-start problem}, has not been explored much by past supervised bug localization studies.

%Our approach
To address the above mentioned limitation, in this work, we propose {\em Deep Transfer Bug Localization} (DTBL) task. DTBL deals with the cold-start problem affecting a target software project by adapting data from other projects. We propose the first DTBL solution, namely \TRANPCNN, which combines deep and transfer learning to address the cold-start problem. \TRANPCNN first extracts the transferable latent features from bug reports and source code files of source and target projects, and then these features are leveraged to generate project-specific predictions for localizing bugs for both source and target projects.  

%\ml{Changed to the model name to TRANP-CNN}

%Novelty over prior work (TCA+, etc.)
There have been a few transfer learning solutions designed to help with cold-start problem in software engineering context. For example, Turhan et al.~\cite{TurhanMBS09} proposed Burak Filter to select $k$ instances from source project that are most similar to the target domain when constructing a defect prediction model for the target project. Nam et al.~\cite{Nam2013transfer} proposed another transfer approach called TCA+ , which maps source and target domain data into a latent space using unsupervised component analysis for cross-project defect prediction. Our approach is unique from previous solutions in the following ways. First, our approach is the first approach ever addressing the cross-project bug localization problem, while the previous approaches are designed for cross-project defect prediction. Second, our approach relies on a deep transfer learning model \TRANPCNN for cross-project bug localization, while all the previous solutions rely on shallow models. Third, our solution is an end-to-end solution that takes a bug report and a source code in their raw format as input and directly output the bug localization result. On the other hand, previous solutions, either the one based on relevant instance selection or latent space construction, consist of multiple steps, where subsequent steps can only work based on the results from preceding steps, even if the results are not suitable for the subsequent steps. 



%employs two convolutional neural network to extract semantic features for bug localization; Second, TRANP-CNN employs transferable feature extraction layers to improve bug localization performance; Third, TRANP-CNN can fully exploit the advantage in using the labeled data from target project, while TCA only uses the distribution of target domain for transfer task in an unsupervised manner. In addition, NP-CNN proposed by Huo et al.~\cite{huo2016learning} has shown good performance in bug localization by introducing particular framework to learn unified features from bug reports and source code. TRANP-CNN is unique from NP-CNN in the way that TRANP-CNN employs a project-specific prediction layer to apply two fully-connected network to train predictors separately from source project and target projects, which enjoy the advantage in extracting the high-level semantic features with the same deep model and training the prediction networks using different models to overcome the inconsistency data distribution. \dl{Please compare the approach with existing work and highlight its novelty.}

%Experiment results

We evaluate the effectiveness of \TRANPCNN on 6 cross-project bug localization tasks involving well-known open source projects. Our experimental results highlight the need for DBTL as existing solutions cannot effectively make use of data from other projects to create a model that can accurately locate bugs in a target project. The results also show that \TRANPCNN outperforms previous state-of-the-art bug localization methods on all 6 tasks across all evaluation measures. In addition, the results highlight the effectiveness of the key components of \TRANPCNN, i.e., the transferable feature extraction layer and the project-specific prediction layer.

%List of contributions
Our contributions are as follows:

\begin{enumerate}

\item We present a new direction of research on cross-project bug localization. We highlight that existing supervised bug localization techniques are unable to perform well when they are trained using data from other projects. 
    
\item We propose a novel deep transfer learning model named \TRANPCNN which learns a transferable latent features shared by both source and target project, and generate project-specific prediction to facilitate a supervised knowledge transfer from the source project to the target project.


% transfer modelemploys programming language specific convolutional neural network to extract transferable semantic features, and a novel heterogeneous predicting adaptation layers are designed to improve cross-project bug localization performance.  

\item Experiments on the well-known open source projects indicate that \TRANPCNN is capable of leveraging rich information from the source project and limited information from the target project to achieve a significantly better performances than the state-of-the-art approaches in terms of Top-k rank, MAP and MRR, suggesting that \TRANPCNN is effective for cross-project bug localization.


\end{enumerate}

%Structure of the paper
The remainder of this paper is as follows. In Section~\ref{sec.background} we summarize the state-of-the-art work on supervised bug localization that our approach builds upon. We elaborate the details of our approach in Section~\ref{sec.approach}. The results of the evaluation of the approach are presented in Section~\ref{sec.exp}. We discuss pertinent points and threats to validity in Section~\ref{sec.discuss}. We describe related work in Section~\ref{sec.related}, before concluding and mentioning future work in Section~\ref{sec.conclusion}. 



\section{Background}\label{sec.background}
In this section, we give a brief introduction about state-of-the-art supervised bug localization NPCNN (Natural and Programming language Convolutional Neural Network), which was proposed by Huo et al.~\cite{huo2016learning}. Our model is built on NPCNN for cross-project bug localization.

The goal of supervised bug localization is training prediction model using bug reports and source code,  and then predicts the localization of buggy code that produces the program behaviors specified in a given bug report. Let $\mathcal{C} =\{ c_1, c_2, \cdots, c_{n_1} \}$ denotes the set of source code , and $\mathcal{R} =\{ r_1, r_2, \cdots, r_{n_2}\} $ denotes the bug reports, where $n_1, n_2$ denote the number of source files and bug reports from source project and target project, respectively. We formulate cross-project bug localization as a learning task which aims to learn a prediction function $f: \mathcal{R} \times \mathcal{C} \mapsto \mathcal{Y}$. $y_{ij} \in \mathcal{Y} = \{+1, -1 \}$ indicates whether a source code $c_j \in \mathcal{C} $ is relevant to a bug report $r_i \in \mathcal{R}$.

Noticing that the semantics of bug reports in natural language and source code in programming language is different, so the NPCNN model employs different convolutional neural networks to semantic features from bug reports and source code, separately. The general structure of NPCNN is illustrated in fig.~\ref{fig:npcnn-structure}. The bug reports and source code are firstly encoded as feature vectors by one-hot algorithm to feed into the CNNs. In the intra-language feature extraction layers, two CNNs are employed for semantic feature extraction: CNN for natural language is followed by the standard approach~\cite{kim2014convolutional}, and CNN for programming language is specifically designed. 

\begin{figure}[hbt]
\centering
\includegraphics[width = \columnwidth]{pic/NPCNN-structure.pdf}
\caption{The general structure of Natural and Programming language Convolutional Neural Network.}
\label{fig:npcnn-structure}
\end{figure}

Huo et al.~\cite{huo2016learning} found that programming language, although in textual format, differs from natural language mainly in two aspects. First, the basic language component carrying meaningful semantics in natural language is word or term, and in programming language the basic language component carrying meaningful semantics is statement. Second, natural language organizes words in a ``flat'' way while programming language organizes its statements in a ``structured'' way to produce richer semantics. Therefore, the structure of CNN for programming language is specifically designed to solve these two points. The first convolutional and pooling layers extract features within statements while preserving the integrity of statements by sliding convolutional window within statements. The subsequent convolutional and pooling layers extract features between statements reflecting the structural nature by employing different size of convolutional windows. More details can be referred in~\cite{huo2016learning}.

\begin{figure}[hbt]
\centering
\includegraphics[width = \columnwidth]{pic/NPCNN.pdf}
\caption{The structure of convolutional neural network for programming language.}
\label{fig:npcnn}
\end{figure}

After feature extraction, the middle-level features generated from bug reports and source code are fed into the cross-language feature fusion layers. To deal with the imbalance nature of bug localization data, the cross-language feature fusion layers introduce an unequal misclassification cost according to the imbalance ratio and train the fully connected network in a cost sensitive manner. Let $cost_n$ denote the cost of incorrectly associate an irrelevant source code file to a bug report and $cost_p$ denote the cost of missing a buggy source code file that is responsible for the reported bugs. Then the weights of the fully connected networks w can be learned by minimizing the following objective function based on SGD (stochastic gradient descent).
\begin{equation}
\begin{aligned}
\label{eq:cost2}
\mathop{\min}_{\mathbf{w}}\sum_{i,j}{[cost_n L(\mathbf{z}^{r}_i, \mathbf{z}^{c}_j, y_{ij}; \mathbf{w})(1-y_{ij})} \\
 {+cost_p L(\mathbf{z}^{r}_i, \mathbf{z}^{c}_i, y_{ij}; \mathbf{w})(1+y_{ij})]}+\lambda||\mathbf{w}||^2
\end{aligned}
\end{equation}


	
\section{Proposed Approach}\label{sec.approach}

\begin{figure*}[hbt]
\centering
\includegraphics[width = 2\columnwidth]{pic/structure.pdf}
\caption{The overall structure of Transfer Natural and Programming language CNN.  The left part is the training process of \TRANPCNN based on the bug reports and source code from source projects and a few data from target projects, the weights of which are trained by minimizing the loss of ensemble loss from fully-connected networks $fc_s$ and $fc_t$. The right part is the testing process, a new bug report and its candidate source code are fed into the model, and \TRANPCNN outputs their relevant scores for bug localization.}
\label{fig:structure}
\end{figure*}

In cross-project bug localization, we are provided with a \emph{source} project with many bug reports carefully localized to the corresponding source code, and a \emph{target} project in which only several bug reports are localized. The goal is to construct a model to fully leverage rich information from the source project to facilitate effective bug localization for the target project.

Let $\mathcal{C}^s = \{ { c^s_1, c^s_2}, \cdots, c^s_{n^c_1} \} $ and $\mathcal{C}^t =\{ c^t_1, c^t_2, \cdots, c^t_{n^c_2} \}$ denote the set of source code files from the source project and the target project, respectively, and $\mathcal{C}=\mathcal{C}^s \bigcup \mathcal{C}^t $. Let $\mathcal{R}^s =\{r^s_1, r^s_2, \cdots, r^s_{n^r_1}\}$ and $\mathcal{R}^t =\{ r^t_1, r^t_2, \cdots, r^t_{n^r_2}\}$ denote the set of bug reports from the source project and target project, respectively, and $\mathcal{R}=\mathcal{R}^s \bigcup \mathcal{R}^t $. In the above notations, $n^c_1, n^c_2, n^r_1, n^r_2$ denote the number of source files and bug reports from source project and target project, respectively. We formulate cross-project bug localization as a learning task which aims to learn prediction functions $\mathbf{f}=(f^s,f^t)$, where $f^\alpha: \mathcal{R}^\alpha \times \mathcal{C}^\alpha \mapsto \mathcal{Y}^\alpha$. $y^\alpha_{ij} \in \mathcal{Y}^\alpha = \{+1, -1\}$ indicates whether a source code $c^\alpha_j \in \mathcal{C}^\alpha $ is relevant to a bug report $r^\alpha_i \in \mathcal{R}^\alpha$, and $\alpha \in \{s,t\}$.   % ******  typos corrected by Ming

Source and target projects may differ from each other in the way how a bug report is localized to the corresponding source code files. To effectively learn the prediction function for a target project, one problem should be addressed carefully: how to identify information from a source project that is potentially useful for learning the prediction function for the target project? Intuitively, if information from the source project can be used for the target project, both must share latent commonalities. Differences observed in the two projects are due to the way these latent commonalities are manifested. Therefore, we argue that the construction of a model that facilitates an effective cross-project bug localization can be decomposed into the two consecutive steps: 1) learning a shared latent feature representation from both source and target project, and 2) biasing the learner towards specific project considering the shared latent feature representation.   % ******  Eidited a little bit by Ming

We realize the aforementioned idea by proposing a novel deep transfer neural network named \TRANPCNN (TRAnsfer Natural and Program Language Convolutional Neural Network). Firstly, \TRANPCNN takes bug reports and source code files as inputs and learns a common transferable latent feature representation shared by both source and target projects. Next, \TRANPCNN  creates a pair of prediction functions that are biased towards the source and target project, respectively, based on the shared feature representation. 

Note that each prediction function are jointly learned with the shared transferable latent feature representation, and thus, the learning of the prediction function for either the source project or the target project is eventually beneficial for learning of the transferable features.  By employing such learning strategy, \TRANPCNN can simultaneously leverage (1) the sufficiently large amount of well-localized data from the source project to help learning better transferable latent features, and (2) the limited number of well-localized data from the target project to adapt the shared transferable features for effective bug localization for the target project. In such a way,  data insufficiency issue in the target project is mitigated by leveraging sufficient number of localized bug reports in the source project.  % ******** Added

\ml{David and Ferdian, please help to read this part to see whether you have got the idea of learning prediction function for the source project.}




\subsection{Model Structure}

The model structure of \TRANPCNN is depicted in Figure~\ref{fig:structure}, where the subfigure to the left depicts the training process of \TRANPCNN, and the one to the right depicts the corresponding test process.

%I do not fully understand these phrases and at the moment I put them here:
%such that the common knowledge shared by both source project and the target project which can eventually facilitate the identification of the correlation between the bug report and the source codes;
%based on the fitted correlations in the source domain and the target domain, respectively

\TRANPCNN consists of four layers: input layer, transferable feature extraction layer, project-specific correlation fitting layer and output layer. The input layer takes bug reports as well as the source code files in their original formats and generate their corresponding encodings such that they can be further processed by the subsequent layers of \TRANPCNN. The transferable feature extraction layer aims to learn an intermediate latent feature representation from the bug reports and source code files shared by source and target projects.  The project-specific prediction layer is responsible for biasing learning, based on the transferable feature representation learned in the previous layer, to identify project-specific correlation patterns between bug reports and source code files in source and target projects. The output layer generates the final correlation scores for report-file pairs (i.e., pairs of bug reports and source code files). It is obvious that the the transferable feature extraction layers and project-specific prediction layer layer are the key parts of the proposed model, which would be explained in details in the following subsections. %******* Edited a little bit by Ming

\ml{Now we use ``project-specific prediction layer". But sometimes, prediction is the final output, and here we in fact output the predicted correlation scores. Other alternative here maybe ``project-specific correlation identification layer" or ``project specific localization generation layer''. Which one do you think is better?}

To train \TRANPCNN model, pairs of bug reports and source code files from the source and target projects along with their ground truth labels (i.e., correlated or not) are fed into the proposed deep model in order to learn the transferable latent feature representation shared by both source and target projects as well as the project-specific prediction functions for the source and target project, respectively. After the model is fully trained, prediction function $f^{t}$ would be used for determine the correlation of each report-file pair $(r_i^t, c_j^t)$ from the target project.

\subsection{Transferable Feature Extraction Layer}

When provided with bug reports and source code files from source and target projects in their original format, the encodings for the bug reports and the source codes are first generated by the input layer. Traditional TF-IDF (term frequency - inverse document frequency) representation~\cite{christopher2008introduction} fails to capture correlation between terms. Thus, we employ word2vec~\cite{abs-1301-3781} encoding to represent both bug reports and source code files with the purpose of enriching the initial representation, based on which, the transferable features are further extracted.

The transferable features for bug reports and source codes files should satisfy the following properties. First, the transferable features should be able to represent the functional semantics in both bug reports and source code files such that the semantics can be further utilized to identify the correlation patterns between the reports and the files. Second, the extracted semantics should be able to capture some common knowledge between the source and target project such that knowledge learned from the source project can eventually be transferred to facilitate learning for the target project.

To allow for effective knowledge transfer between source and target projects, a key challenge here is how to identify what information is useful and transferable and what is useful but may not be transferable. To address this challenge, we employ a special strategy we refer to as \emph{weight sharing} during the learning process, which imposes a hard constraint that the learned weights in the networks (both N-CNN and P-CNN) for the target project should be exactly the same as the learned weights in the networks for the source project. By weight sharing, the learning procedure that optimizes for good bug localization performances on both source project and target project is forced to focus on common features shared by both source and target projects rather than extracting project-specific features for each project. Thus, the resulting features allow for \emph{transfer} of common knowledge that is useful to determine correlation between bug reports and source code files from the source project to the target project.
 

%I'm not sure of the following sentence but the paragraph seems okay without it ... I'm not sure ...
%Note that such knowledge may need a large amount of labeled training data to be learned for any single project, but now can be directly used for the target project where the labeled training data may be limited especially for the code start.

\subsection{Project-Specific Prediction Layer} % The whole sectin is revised.

After being processed by the transferable feature extraction layers, bug reports (as well as the source code files) from both source and target project are represented using the \emph{same} set of transferable features embedded with common knowledge shared by both source and target projects.


However, since the source project and target projects may differ in the way how bug reports and source code files are correlated, the subsequent project-specific prediction layer is responsible to learn the project-specific correlation patterns considering the same set of transferable features.

Specifically, we utilize two fully-connected networks sharing the same input from the transferable feature extraction layer, namely FCN$^s$ and FCN$^t$, to learn the correlation patterns between the bug report and the source code files for both source and target projects, based on which the final prediction of correlation score is generated in the proceeding output layer. As mentioned before, learning the correlation patterns for the source project can leverage the rich supervision of the localized report-file pairs in the source project in helping the learning good transferable features, which can be conceptually regarded as pushing the useful knowledge for localizing bugs from the source project down to the transferable latent features and transferred to the target project when learning correlation patterns for the target project based on the shared latent features. 

\dl{Why do we want simultaneously learn two prediction functions? Why not just learn for target project? I'm not fully sure here ...}
\ml{Hopefully the previous paragraph addresses david's conerns.}

To jointly learn the project-specific correlation patterns for each project along with the transferable latent features, we propose the a hybrid loss function that simultaneously fulfill the learning objectives for both source and target projects. 

%Let $\varphi^{s}_{i}$ and $\psi^{s}_{j}$ denote the transferable features learn from a report-file pair $(r^s_i, c^s_j)$ with its correlation label $y^s_{ij}$ from the source project and $\varphi^{t}_{i}$ and $\psi^{t}_{j}$ with its correlation label $y^t_{ij}$ from the target project, respectively. Let $\mathbf{W} = [W_{CNN}, W_{FCN}}$ denote all the weights to be learned for the entire




%$\mathcal{L}$ is the square loss, $\lambda$ is the trade-off parameter and the weight vectors $W$ contains the weight vectors in convolutional neural networks $W_{conv}$, in fully-connected network of source domain $W_{fc_s}$ and in fully-connected network of target domain $W_{fc_t}$.

%The key challenge here is how to simultaneously learn two prediction functions capturing the correlation patterns in different project based on the same set of features within the same model structure. To address this challenge, we construct two fully connected neural network substructures, namely $fc^s$ and $fc^t$, for the subsequent project-specific correlation fitting layers, where $fc^s$ and $fc^t$ share the same input unit and have different output units. Such a special design facilitates the model to take the same set of transferable features from both bug reports and source code files and fit to correlation pattern of the report-code pair in the source project and target project, respectively, with different substructures $fc^s$ and $fc^t$.

\dl{The following phrase is unclear to me: fit to correlation pattern of the report-code pair in the source project and target project, respectively, with different substructures $fc^s$ and $fc^t$.}
\dl{I'm not sure how to fit network substructures to correlation patterns? Do we mean we use the substructures to *learn* different correlation patterns?}


\begin{equation}
\begin{aligned}
\label{eq:lossfunction}
\mathop{\arg\min}_{\mathbf{W}}&\sum_{i,j}\mathcal{L}(r^s_i,c^s_j,,y^s_{ij}; \mathbf{W}) \\
+&\sum_{i,j}\mathcal{L}(r^t_i,c^t_j,y^t_{ij}; \mathbf{W})+\lambda||\mathbf{W}||^2
\end{aligned}
\end{equation}
%
where $\mathbf{W} = [W_{CNN}, W_{FCN}]$ is the vector all the weights on the network connections to be learned for the entire \TRANPCNN, and $\mathcal{L}$ is the any widely used loss function for deep learning that computes the empirical loss of the network with weights $\mathbf{W}$ by taking the report-code pair $(r^s_i, c^s_j)$ and $(r^t_i, c^t_j)$ from the source and target project as well as their correlation label $y^s_{ij}$ and $y^t_{ij}$, respectively. The first term evaluates how well the prediction is for the source project, the second terms evaluates how well the prediction for the target project, and the third term is an regularization term balanced by a hyper-parameter $\lambda$ in the purpose of avoid the model to be learned going to the wild.


We train \TRANPCNN based on back-propagation by minimizing the hybrid loss function using using the stochastic gradient descent (SGD) \cite{duchi2011adaptive}.

%\lambda is a hyepr-parameter that balance the empirical loss and the regularized term. Ther

%In the above equation, $\mathcal{L}$ is the square loss, $\lambda$ is the trade-off parameter and the weight vectors $W$ contains the weight vectors in convolutional neural networks $W_{conv}$, in fully-connected network of source domain $W_{fc_s}$ and in fully-connected network of target domain $W_{fc_t}$.

%\dl{The intuition for the above function may need to be elaborated}

%By solving this optimization problem using the stochastic gradient descent (SGD), the proposed \TRANPCNN can be effectively learned.

%\dl{Need to cite SGD.}
%\dl{I'm not sure if we learn the whole TRANPCNN using SGD or only some parts of it. If it is some parts of it, which parts are not mentioned in the description above.} 

\section{Experiments}\label{sec.exp}
To evaluate the effectiveness of \TRANPCNN, we conduct experiments on open source software projects and compare it with several state-of-the-art bug localization methods.

\subsection{Research Questions}
%\dl{Ferdian, please add details on why the RQs are interesting and the purpose of the RQs}

Our experiments are designed to address the following research questions:

\vspace{0.2cm}\noindent\textit{\textbf{RQ1}: Is there a need for a specialized technique for cross-project bug localization?}

If a model learned from one project can be used for others project, then there is no need for a specialized technique for cross-project bug localization. Thus, before we consider other research questions, we validate the need for our proposed approach by empirically evaluating the effectiveness of a model learned from one project to localize bugs in other projects.

\vspace{0.2cm}\noindent\textit{\textbf{RQ2}: Does the cross-project feature fusion layer improve the bug localization performance?}

\dl{Xuan, please help to motivate this research question. I can't motivate it since I believe the details of the approach is going to change.}

In Section 3, we propose to employ heterogeneous predicting layers adaptation that apply two fully-connected networks for prediction from source and target projects separately, which is the key part of TRANP-CNN. In this research question, we evaluate whether the heterogeneous predicting adaptation layers help improve the bug localization performance by comparing the results previous NPCNN and SimpleTrans model. 

\vspace{0.2cm}\noindent\textit{\textbf{RQ3}: Can TRANP-CNN outperform other bug localization methods?}

A number of bug localization methods have been proposed in the literature. In this research question, we evaluate whether and to what extent can our proposed approach TRANP-CNN outperforms the state-of-the-art methods designed for bug localization and those that can be adapted for bug localization. 

\subsection{Datasets}
We consider datasets that were previously studied and manually vetted by Herzig et al.~\cite{HerzigJZ13} and Kochhar et al.~\cite{KochharTL14}. Herzig et al. highlighted that many reports in issue tracking systems are wrongly labelled as bugs when they are actually feature requests (and vice versa). Kochhar et al. highlighted that many bug reports are already {\em fully localized}, i.e., developers have already identified all buggy source code files in the bug reports. For such bug reports, bug localization tool is no longer needed.

To deal with these biases, Herzig et al. have manually classified 5,591 issue reports from JIRA issue tracking systems of three projects (HTTPClient, Jackrabbit, and Lucene-Java) and 1,810 issue reports from Bugzilla issue tracking systems of two projects (Rhino and Tomcat 5) and released the dataset publicly\footnote{\url{https://www.st.cs.uni-saarland.de/softevo/bugclassify}}. Kochhar et al. have analyzed 1,191 reports from JIRA issue tracking systems that were confirmed by Herzig et al. as bug reports. They focused on reports from JIRA issue tracking systems since a number of studies have shown that reports in Bugzilla are often poorly linked with commits that fix them~\cite{BachmannBRDB10,BirdBADBFD09}, while bug reports in JIRA are often more well-linked as JIRA provides a utility to better connect bug reports to their corresponding commits~\cite{BissyandeTWLJR13}. Kochhar et al. have identified a set of 398 bug reports that are already {\em fully localized} and another set of 793 bug reports that are either {\em partially localized} (i.e., reports where some of the files containing the bugs are explicitly mentioned in the report) or {\em not localized} (i.e., reports which do not explicitly specify any of the buggy files.). They also have released this dataset publicly\footnote{\url{https://github.com/smusis/buglocalizationbiases}}.

In this work, we consider the 793 bug reports from three software systems: 63 from HTTPClient (H), 534 from Jackrabbit (J), and 196 from Lucene-Java (L), which are provided by Kochhar et al. HTTPClient\footnote{\url{http://hc.apache.org/httpcomponents-client-ga/index.html}} is a library for implementing the client side of HTTP standard, while Jackrabbit\footnote{\url{https://jackrabbit.apache.org/jcr/index.html}} is a content repository, and Lucene is a text search engine\footnote{\url{ http://lucene.apache.org/}}. The details of the reports considered in this study are shown in Table~\ref{tab:reports}. Although the number of reports considered is fewer than those considered in several prior work, these reports are of high-quality. They has been manually vetted before and are absent from well-known biases identified by Herzig et al. and Kochhar et al. Most past studies have ignored these well-known biases and thus introduce a threat to the validity of their findings.

%that have been fully localized by developers. They focus on bug reports
%containing a total of 5,591 reports from JIRA issue tracking systems of
%We only consider projects with JIRA issue tracking systems since links between bug reports and their bug fixing commits stored in them are typically more reliable than those stored in Bugzilla issue tracking systems -- c.f.,~\cite{BissyandeTWLJR13}. The details of the reports considered in this study are shown in Table~\ref{tab:reports} and they have been used before by Kochhar et al.~\cite{KochharTL14}.

%\begin{table}
%\caption{Bug Report Dataset}\label{tab:reports}
%\begin{tabular}{|l|l|l}
%\hline
%{\bf Project} & {\bf \# Reports} \\
%\hline HTTPClient & 63\\
%\hline Jackrabbit & 534\\
%\hline Lucene-Java & 196\\
%\hline
%\end{tabular}
%\end{table}

%\dl{Xuan, I thought we are also using your previous datasets? If we only use Pavneet's dataset, do we use all of them or only some of them that are not biased? Pavneet showed that some of the dataset is biased ... reviewers may be concerned with this bias ... Bias 2 mentioned in this paper (\url{http://ink.library.smu.edu.sg/cgi/viewcontent.cgi?article=3425&context=sis_research}) is particularly important.}

%\xh{Currently we only use Paveneet's datasets (H,J and L), and yes we have considered the bias, and we only use the unbiased data sets. The ``fully localized'' bug reports are filtered. Our previous data sets are bias, so we are not sure if we use here is suitable.  If necessary, we can conduct several comparison experiments ono more data sets. }

%\dl{Ferdian, please add details datasets. Please see the following papers for details of the datasets:~\cite{huo2016learning,KochharTL14}.}



\subsection{Evaluation Metrics}

The evaluation metrics are presented here.

\dl{Ferdian, please add details on evaluation metrics.}



\subsection{Baselines}

We compare our proposed model \TRANPCNN with following baseline methods:
\begin{itemize}
  \item VSM (Vector Space Model)~\cite{rao2011retrieval}: a baseline method that firstly uses Vector Space Model to represent the text bug reports and source code, then employs Logistic Regression to predict the related buggy source code.
  \item Burak (Burak Filter)~\cite{peters2013better}: a state-of-the-art method for cross-project and cross-company defect prediction problem, which filters training sets using Burak filter that employs k-nearest neighbour to selects instances in the source project similar to the test project.
  \item TCA+ (Transfer Component Analysis)~\cite{NamPK13}: a state-of-the-art transfer learning method in software engineering, which firstly normalizes the data and employs TCA to map source and target project into a same feature space and then apply Logistic Regression for bug localization (same settings suggested in their paper). 
  \item TCA+$^{(P)}$ (Transfer Component Analysis with multi-layer Perceptron): a state-of-the-art transfer learning method in software engineering, which firstly normalizes the data and employs TCA to map source and target project into a same feature space and then apply MLPs for bug localization (same settings with fully-connected layers in \TRANPCNN).
   \item TCA+$^{(D)}$ (Transfer Component Analysis with Deep features): a state-of-the-art transfer learning method in software engineering, which firstly normalizes the data and employs TCA to map source and target project into a same feature space and then apply Logistic Regression for bug localization (using deep features extracted from CNN instead of VSM features).
  \item NP-CNN (Natural and Programming language Convolutional Neural Network)~\cite{huo2016learning}: a state-of-the-art deep model for bug localization, which uses source project data for training a deep convolutional neural network and localizes the buggy source code for target project data.
\end{itemize}
%\ft{Settings would probably better be grouped in experimental settings section. There is no citation for TCA. }

\subsection{Experimental Settings}
First, the parameter settings of baseline methods (VSM, burak, TCA+, TCA+$^{(P)}$, TCA+$^{(D)}$), we use the same parameters settings suggested in their work~\cite{rao2011retrieval}\cite{NamPK13}. The parameters are set the same in~\cite{huo2016learning}. 

For the \TRANPCNN model, we employ the most commonly used ReLU $\sigma(x)=\max(0,x)$ as active function and the filter windows size $d$ is set as 3, 4, 5, with 100 feature maps each in Within-Project feature extraction layers. The number of neurons in fully-connect network in  set the same number of CNN. In addition, the drop-out method is also applied which is used to prevent co-adaption of hidden units by randomly dropping out values in fully-connected layers, and the drop-out probability $p$ is set 0.25 in our experiments.

For data partition, we use data from source projects and 20\% target projects as training sets, and locates the 80\% buggy code in target projects. This process repeats for 5 times to reduce the influence of randomness, and we report the average results for comparison. 

\section{Experimental Results}

\subsection{Experimental Results for Research Questions}

\begin{table}[htbp]
  \centering
  \caption{Performance Comparisons between within-project and cross-project bug localization.}
  \resizebox{!}{0.5\columnwidth}{
    \begin{tabular}{c|l|c|c|c|c|c}
    \toprule
    Tasks & \textit{Methods} & \multicolumn{1}{l}{\textit{Top 1}} & \multicolumn{1}{l}{\textit{Top 5}} & \multicolumn{1}{l}{\textit{Top 10}} & \multicolumn{1}{l}{\textit{MAP}} & \multicolumn{1}{l}{\textit{MRR}} \\
    \midrule
    \multirow{3}[0]{*}{\textbf{J}$\rightarrow$\textbf{H}} & NPCNN & 0.317  & 0.362  & 0.508  & 0.276  & 0.352  \\
          & NPCNN$^{partial}$ & 0.204  & 0.258  & 0.313  & 0.202  & 0.292  \\
          & NPCNN$^{full}$ & \textbf{0.533}  & \textbf{0.617}  & \textbf{0.650}  & \textbf{0.472}  & \textbf{0.580}  \\
          \midrule
    \multirow{3}[0]{*}{\textbf{L}$\rightarrow$\textbf{H}} & NPCNN & 0.142  & 0.192  & 0.345  & 0.161  & 0.218  \\
          & NPCNN$^{partial}$ & 0.204  & 0.258  & 0.313  & 0.202  & 0.292  \\
          & NPCNN$^{full}$ & \textbf{0.533}  & \textbf{0.617}  & \textbf{0.650}  & \textbf{0.472}  & \textbf{0.580}  \\
          \midrule
    \multirow{3}[0]{*}{\textbf{H}$\rightarrow$\textbf{J}} & NPCNN & 0.167  & 0.287  & 0.349  & 0.247  & 0.277  \\
          & NPCNN$^{partial}$ & 0.035  & 0.211  & 0.302  & 0.155  & 0.189  \\
          & NPCNN$^{full}$ & \textbf{0.508}  & \textbf{0.587}  & \textbf{0.679}  & \textbf{0.462}  & \textbf{0.557}  \\
          \midrule
    \multirow{3}[0]{*}{\textbf{L}$\rightarrow$\textbf{J}} & NPCNN & 0.152  & 0.182  & 0.318  & 0.176  & 0.221  \\
          & NPCNN$^{partial}$ & 0.035  & 0.211  & 0.302  & 0.155  & 0.189  \\
          & NPCNN$^{full}$ & \textbf{0.508}  & \textbf{0.587}  & \textbf{0.679}  & \textbf{0.462}  & \textbf{0.557}  \\
          \midrule
    \multirow{3}[0]{*}{\textbf{H}$\rightarrow$\textbf{L}} & NPCNN & 0.173  & 0.246  & 0.390  & 0.196  & 0.329  \\
          & NPCNN$^{partial}$ & 0.097  & 0.219  & 0.335  & 0.095  & 0.109  \\
          & NPCNN$^{full}$ & \textbf{0.289}  & \textbf{0.484}  & 0.611  & \textbf{0.287}  & \textbf{0.387}  \\
          \midrule
    \multirow{3}[0]{*}{\textbf{J}$\rightarrow$\textbf{L}} & NPCNN & 0.110  & 0.255  & 0.323  & 0.141  & 0.176  \\
          & NPCNN$^{partial}$ & 0.097  & 0.219  & 0.335  & 0.095  & 0.109  \\
          & NPCNN$^{full}$ & \textbf{0.289}  & \textbf{0.484}  & \textbf{0.611}  & \textbf{0.287}  & \textbf{0.387}  \\
          \bottomrule
    \end{tabular}%
    }

  \label{tab:results1}%
\end{table}%


\begin{figure}[hbt]
\centering
\includegraphics[width = 0.9\columnwidth]{pic/results1_avg.pdf}
\caption{Performance comparisons between within-project and cross-project bug localization.}
\label{fig:results1}
\end{figure}


\textbf{RQ1}: \textit{Is there a need for cross-project bug localization?}

To answer this research question, we compare the results of using NP-CNN for bug localization in different settings.

\begin{itemize}
  \item NPCNN: Employ NPCNN directly for cross-project bug localization, which means directly training the model on the data from source projects and locating the bugs in the target project.
  \item NPCNN$^{partial}$: Employ NPCNN using partial data of target projects, which means training based on a few data (20\%) in the target projects, and localizes target buggy files without using data from source project.
  \item NPCNN$^{full}$: Employ NPCNN using full data of target projects. In this setting, we conduct 5-folds cross-validation for comparison.
\end{itemize}

The results are detailed in Table.~\ref{tab:results1} and Figuire.~\ref{fig:results1}. There are six tasks in the table, in which $\textbf{H}$ represents project \textit{HTTPClient}, $\textbf{J}$ represents project \textit{Jackrabbit} and $\textbf{L}$ represents \textit{Lucene-Java}. Meanwhile, the task $\textbf{H} \rightarrow \textbf{J}$ represents using \textit{HTTPClient} as source project and predicts the location of buggy files in target project \textit{Jackrabbit}. The results show that the performance of bug localization using full data of target projects is the best, which has a large gap against the performance using partial data. For cross-project bug localization, the performance of NPCNN that directly uses source projects is better than NPCNN$^{partial}$, showing that cross-project data is beneficial to improve the bug localization performance, but directly using within-project bug localization technique will not as well as NPCNN$^{full}$. The results suggest that there is a need for cross-project bug localization, and directly using within-project bug localization method does not show good performance.

\textbf{RQ2}: \textit{Do the project-specific correlation fitting layers improve the bug localization performance?}

To answer this research question, we compare the results of \TRANPCNN with the original NPCNN. The difference of the structure between \TRANPCNN and NPCNN is that \TRANPCNN employs two fully-connected networks to combine deep features from source projects and target projects in the project-specific correlation fitting layers, respectively, which will counter the influences that cross-project data may have different distribution leading to a bias performance. The results are detailed in Tab.~\ref{tab:results2}.

%\ft{It would be good if we can explain why such structure can counter bias. Perhaps we should explain this first in approach (theory) and then highlight it again accompanied with experimental result (empirical).}

\begin{table}[htbp]
  \centering
  \caption{Performance comparisons with previous deep models.}
  \resizebox{!}{0.35\columnwidth}{
    \begin{tabular}{c|l|c|c|c|c|c}
    \toprule
    Tasks & \textit{Methods} & \textit{Top 1} & \textit{Top 5} & \textit{Top 10} & \textit{MAP} & \textit{MRR} \\
    \midrule
    \multirow{2}[0]{*}{\textbf{J}$\rightarrow$\textbf{H}} & NPCNN & 0.317  & 0.362  & 0.508  & 0.276  & 0.352  \\
%          & SimpleTrans & 0.354 & 0.396 & 0.563 & 0.298 & 0.395 \\
          & \TRANPCNN & \textbf{0.500}   & \textbf{0.583} & \textbf{0.625} & \textbf{0.376} & \textbf{0.543} \\
          \midrule
    \multirow{2}[0]{*}{\textbf{L}$\rightarrow$\textbf{H}} & NPCNN & 0.142  & 0.192  & 0.345  & 0.161  & 0.218  \\
%          & SimpleTrans & 0.163 & 0.146 & 0.354 & 0.141 & 0.246 \\
          & \TRANPCNN & \textbf{0.275} & \textbf{0.35}  & \textbf{0.488} & \textbf{0.242} & \textbf{0.332} \\
          \midrule
    \multirow{2}[0]{*}{\textbf{H}$\rightarrow$\textbf{J}} & NPCNN & 0.167  & 0.287  & 0.349  & 0.247  & 0.277  \\
%          & SimpleTrans & 0.133 & 0.324 & 0.365 & 0.273 & 0.301 \\
          & \TRANPCNN & \textbf{0.396} & \textbf{0.443} & \textbf{0.514} & \textbf{0.371} & \textbf{0.434} \\
          \midrule
    \multirow{2}[0]{*}{\textbf{L}$\rightarrow$\textbf{J}} & NPCNN & 0.152  & 0.182  & 0.318  & 0.176  & 0.221  \\
%          & SimpleTrans & 0.144 & 0.204 & 0.382 & 0.247 & 0.249 \\
          & \TRANPCNN & \textbf{0.460}  & \textbf{0.462} & \textbf{0.488} & \textbf{0.404} & \textbf{0.478} \\
          \midrule
    \multirow{2}[0]{*}{\textbf{H}$\rightarrow$\textbf{L}} & NPCNN & 0.173  & 0.246  & 0.390  & 0.196  & 0.329  \\
%          & SimpleTrans & 0.197 & 0.323 & 0.426 & 0.152 & 0.313 \\
          & \TRANPCNN & \textbf{0.361} & \textbf{0.445} & \textbf{0.535} & \textbf{0.279} & \textbf{0.414} \\
          \midrule
    \multirow{2}[0]{*}{\textbf{J}$\rightarrow$\textbf{L}} & NPCNN & 0.110  & 0.255  & 0.323  & 0.141  & 0.176  \\
%          & SimpleTrans & 0.140  & 0.282 & 0.342 & 0.163 & 0.224 \\
          & \TRANPCNN & \textbf{0.301} & \textbf{0.410}  & \textbf{0.517} & \textbf{0.247} & \textbf{0.368} \\
          \bottomrule
    \end{tabular}%
    }
  \label{tab:results2}%
\end{table}%

\begin{figure}[hbt]
\centering
\includegraphics[width = 0.9\columnwidth]{pic/results2_avg.pdf}
\caption{Performance comparisons with deep models.}
\label{fig:results2}
\end{figure}

The results show that \TRANPCNN performs better than NP-CNN, which suggests that the project-specific correlation fitting layers  can improve the performance of cross-projects bug localization. It is because the project-specific correlation fitting layers employ two separate fully-connected network to learn the prediction function for each project separately to encounter the bias of data distribution and the results have provided the evidences.  

\textbf{RQ3}: \textit{Can \TRANPCNN outperform other bug localization methods?}

To answer this research question, we compare the results of \TRANPCNN with state-of-the-art methods: VSM, Burak, TCA+, TCA+$^(P)$. VSM is a baseline technique used in the within-project bug localization and we employ it on cross-project bug localization for comparison. Burak and TCA+ have been shown good performance on cross-project and cross-company defect prediction, and also we apply it on cross-project bug localization. The parameters of VSM and Burak Filter are set suggested in their work. For fair comparison with TCA+, we use same normalization strategy and the classifier in their original paper (Logistic Regression) and multi-layer perception (same as \TRANPCNN) are compared in our experiments. The results are detailed in Tab.~\ref{tab:results3}. 

%\ft{We should directly use abbreviations of baseline names since they have been introduced in Baselines section. Explanations about baselines are also redundant. }

According to the results, we have several findings: 1. Burak and TCA techniques perform better than the baseline VSM model, indicating that using transfer algorithms is able to improve the performance in cross-project bug localization; 2. \TRANPCNN outperforms TCA-P, which shows that the high-level features extracted from CNN are more semantic and informative, leading to a better representation and bug localization performance; 3. TCA-D uses deep features extracted from CNN and the performance is not as well as \TRANPCNN, which further proves that the cross-project feature fusion layers improve bug localization performance; 4. \TRANPCNN obtains the best average values in terms of all evaluation metrics. Comparing to the best baseline TCA+$^{(D)}$, \TRANPCNN improves the results by 24.6\% in terms of Top-1, by 22.6\% in terms of Top-5, by 20.9\% in terms of Top-10, by 21.9\% in terms of MAP and by 17.2\% in terms of MRR. According to Mann-Whitney U-test, we find \TRANPCNN significant better in terms of all metrics. The results suggest that \TRANPCNN outperforms other traditional bug localization methods and transfer techniques on software engineering.

%\ft{We can highlight the amount of improvement that \TRANPCNN achieves as compared to the best baseline.}

\begin{table}[htbp]
  \centering
  \caption{Performance comparisons with traditional bug localization models.}
  \resizebox{!}{0.9\columnwidth}{
    \begin{tabular}{c|l|c|c|c|c|c}
    \toprule
    Tasks & \textit{Methods} & \multicolumn{1}{c|}{\textit{Top 1}} & \multicolumn{1}{c|}{\textit{Top 5}} & \multicolumn{1}{c|}{\textit{Top 10}} & \multicolumn{1}{c|}{\textit{MAP}} & \multicolumn{1}{c}{\textit{MRR}} \\
    \midrule
    \multirow{6}[0]{*}{\textbf{J}$\rightarrow$\textbf{H}} & VSM   & 0.098  & 0.157  & 0.177  & 0.087  & 0.143  \\
          & Burak & 0.110  & 0.126  & 0.138  & 0.116  & 0.121  \\
          & TCA-R & 0.120  & 0.212  & 0.144  & 0.157  & 0.162  \\
          & TCA-P & 0.114  & 0.133  & 0.154  & 0.123  & 0.176  \\
          & TCA-D & 0.122  & 0.225  & 0.271  & 0.168  & 0.248  \\
          & \TRANPCNN & \textbf{0.500}  & \textbf{0.583}  & \textbf{0.625}  & \textbf{0.376}  & \textbf{0.543}  \\
          \midrule
    \multirow{6}[0]{*}{\textbf{L}$\rightarrow$\textbf{H}} & VSM   & 0.059  & 0.098  & 0.237  & 0.099  & 0.112  \\
          & Burak & 0.113  & 0.203  & 0.242  & 0.143  & 0.143  \\
          & TCA-R & 0.120  & 0.188  & 0.244  & 0.151  & 0.158  \\
          & TCA-P & 0.128  & 0.200  & 0.252  & 0.161  & 0.167  \\
          & TCA-D & 0.102  & 0.237  & 0.367  & 0.161  & 0.202  \\
          & \TRANPCNN & \textbf{0.275}  & \textbf{0.350}  & \textbf{0.488}  & \textbf{0.242}  & \textbf{ 0.332}  \\
          \midrule
    \multirow{6}[0]{*}{\textbf{H}$\rightarrow$\textbf{J}} & VSM   & 0.035  & 0.211  & 0.232  & 0.165  & 0.129  \\
          & Burak & 0.130  & 0.150  & 0.206  & 0.225  & 0.195  \\
          & TCA-R & 0.115  & 0.162  & 0.209  & 0.239  & 0.244  \\
          & TCA-P & 0.114  & 0.154  & 0.203  & 0.237  & 0.241  \\
          & TCA-D & 0.111  & 0.135  & 0.157  & 0.168  & 0.185  \\
          & \TRANPCNN & \textbf{0.396}  & \textbf{0.443}  & \textbf{0.514}  & \textbf{0.371}  & \textbf{0.434}  \\
          \midrule
    \multirow{6}[0]{*}{\textbf{L}$\rightarrow$\textbf{J}} & VSM   & 0.197  & 0.212  & 0.293  & 0.167  & 0.216  \\
          & Burak & 0.161  & 0.132  & 0.368  & 0.170  & 0.187  \\
          & TCA-R & 0.136  & 0.183  & 0.370  & 0.170  & 0.179  \\
          & TCA-P & 0.114  & 0.116  & 0.397  & 0.138  & 0.191  \\
          & TCA-D & 0.178  & 0.236  & 0.469  & 0.227  & 0.256  \\
          & \TRANPCNN & \textbf{0.460}  & \textbf{0.462}  & \textbf{0.488}  & \textbf{0.404}  & \textbf{0.478}  \\
          \midrule
    \multirow{6}[0]{*}{\textbf{H}$\rightarrow$\textbf{L}} & VSM   & 0.083  & 0.278  & 0.393  & 0.154  & 0.136  \\
          & Burak & 0.105  & 0.226  & 0.272  & 0.123  & 0.222  \\
          & TCA-R & 0.136  & 0.208  & 0.383  & 0.170  & 0.279  \\
          & TCA-P & 0.143  & 0.226  & 0.394  & 0.171  & 0.288  \\
          & TCA-D & 0.162  & 0.207  & 0.345  & 0.229  & 0.292  \\
          & \TRANPCNN & \textbf{0.361}  & \textbf{0.445}  & \textbf{0.535}  & \textbf{0.279}  & \textbf{0.414}  \\
          \midrule
    \multirow{6}[0]{*}{\textbf{J}$\rightarrow$\textbf{L}} & VSM   & 0.038  & 0.077  & 0.154  & 0.124  & 0.204  \\
          & Burak & 0.138  & 0.161  & 0.176  & 0.168  & 0.226  \\
          & TCA-R & 0.135  & 0.111  & 0.172  & 0.169  & 0.222  \\
          & TCA-P & 0.136  & 0.132  & 0.192  & 0.173  & 0.237  \\
          & TCA-D & 0.142  & 0.297  & 0.308  & 0.238  & 0.293  \\
          & \TRANPCNN & \textbf{0.301}  & \textbf{0.410}  & \textbf{0.517}  & \textbf{0.247}  & \textbf{0.368}  \\

          \bottomrule
    \end{tabular}%
    }
  \label{tab:results3}%
\end{table}%

\begin{figure}[hbt]
\centering
\includegraphics[width = 0.9\columnwidth]{pic/results3_avg.pdf}
\caption{Performance comparisons with traditional bug localization models.}
\label{fig:results3}
\end{figure}



\section{Discussion}\label{sec.discuss}
\subsection{Why do the cross-project feature fusion layers work? }

\subsection{Why does TRANP-CNN improve the bug localization performance?}

\subsection{Threats to Validity}


\section{Related Work}\label{sec.related}
In this section, we first describe existing work on bug localization in Section~\ref{sec.bugloc}. Next, we present existing work that also deal with cold-start problem in software engineering in Section~\ref{sec.crossproj}. Finally, we describe recent effort in software engineering that adapts deep learning to software engineering.

\subsection{Bug Localization}\label{sec.bugloc}

\dl{Ferdian, please identify more related papers and provide their descriptions below. Please also classify the approach into supervised and unsupervised.}

A number of papers have proposed various techniques that take as input a bug report and return a ranked list of source code files that are relevant to it~\cite{lukins2008source,RaoK11,SahaLKP14,rao2013incremental,huo2016learning}. These {\em text-based} bug localization techniques can be divided into two general families: supervised approaches~\cite{huo2016learning} and unsupervised ones~\cite{rao2013incremental}. Supervised approaches learn a model from data of bug reports whose relevant buggy source code files have been identified. Unsupervised approaches do not learn such model. We briefly introduce approaches that belong to each family below.

\vspace{0.2cm}\noindent{\bf Unsupervised Approaches.}

\vspace{0.2cm}\noindent{\bf Supervised Approaches.}

\subsection{Cross-Project Learning}\label{sec.crossproj}

The problem of scarcity of labelled data for a target project (aka. cold-start problem) has been explored in several automated software engineering tasks~\cite{ZimmermannNGGM09,TurhanMBS09,NamPK13,KitchenhamMT07}. Closest to our work, is the line of work on cross-project defect prediction~\cite{ZimmermannNGGM09,TurhanMBS09,NamPK13}. Note that defect prediction does not consider a target bug report, while bug localization takes as input a bug report and return files relevant to it. They are used in different software development phase, i.e., code inspection and testing (defect prediction) vs. debugging (bug localization), and thus are thus complementary with each other. We provide a description of existing work on cross-project defect prediction below.

Zimmermann et al. are among the first to investigate cross-project defect prediction~\cite{ZimmermannNGGM09}. They highlight that defect prediction works well if there is a sufficient amount of data from a project to train a model. However, they argue that sufficient data is often unavailable for many projects (especially new ones) and companies. One way to deal with the problem is to build a model from a project with sufficient data and use the model to predict defective code in another project -- which is referred to as cross-project defect prediction. To investigate viability of cross-project defect prediction, Zimmermann et al. consider 12 target projects and demonstrate that cross-project defect prediction is ``a serious challenge'' -- it is not possible to achieve good results by simply using models built from other projects. 

Zimmermann et al.'s study is a call-to-arms that spur active interest in the area of cross-project defect prediction. A number of solutions have been proposed to boost the effectiveness of cross-project defect prediction. These include the work by Turhan et al.~\cite{TurhanMBS09} and Nam et al.~\cite{NamPK13} highlighted below. 

Turhan et al. propose a relevancy filtering method to select training data that are closest to test data~\cite{TurhanMBS09}. In particular, they employ a k-nearest neighbor method to pick k training instances (i.e., files from a project with known defect labels) that are closest to each test data (i.e., files from a target project with unknown defect labels). The resultant training instances are then used to learn a model that is then applied to predict defect labels of files from the target project in the test data. The approach by Turhan et al. potentially omit many training instances, which may reduce the effectiveness of the resultant model. Nam et al. deal with cross-project defect prediction problem by leveraging the recent development in machine learning -- i.e., transfer learning~\cite{NamPK13}. In particular, they take an existing transfer learning method -- referred to as Transfer Component Analysis (TCA)~\cite{PanTKY11} -- and adapt it for defect prediction.

Following existing cross-project defect prediction studies, we first demonstrate that cross-project bug localization is a serious challenge (see RQ1 in Section~\ref{sec.exp}). We then propose a novel deep transfer learning method to deal with this challenge. We have also compared our solution with several adaptations of Turhan et al.'s relevancy filtering method and TCA~\cite{PanTKY11} for bug localization, and demonstrated that our solution outperforms these baselines.

%\dl{Ferdian, please include some existing work on cross-project defect prediction. One recent work by our group is~\cite{XiaLPNW16}.}

\subsection{Deep Learning in Software Engineering}\label{sec.deeplearning}

Recently, deep learning~\cite{Goodfellow-et-al-2016}, which is a recent breakthrough in machine learning domain, has been applied in many areas. Software engineering is not an exception. Our approach also employs deep learning. Thus, we review related studies that also employ deep learning to improve automated software engineering. In the process, we highlight the difference between our approach and the existing work, and thus stress our novelty.

\dl{Ferdian, please include some existing work on deep learning in SE. Some recent work include the following: \cite{WangLT16},~\cite{YangLXZS15},~\cite{0004CC17},~\cite{LeeHLKJ17},~\cite{XuYXXCL16}}.


\section{Conclusion and Future Work}\label{sec.conclusion}
Recent supervised bug localization techniques make use of a training dataset of manually localized bugs to boost performance. Unfortunately, Zimmermann et al. have highlighted that sufficient defect data is often unavailable for many projects and companies~\cite{ZimmermannNGGM09}. Much defect data is also not clean and suffer from a variety of biases -- c.f.,~\cite{HerzigJZ13,KochharTL14,BachmannBRDB10,BirdBADBFD09}. To deal with this limitations, there is a need for a cross-project bug localization solution that can take data from a project to train a model for another project for which only limited clean bug data is available. To address this need, we propose a deep transfer learning approach specialized for bug localization named \TRANPCNN. \dl{Please add a sentence that remind readers about the novelty of the approach.} Experiments on manually curated datasets by Herzig et al.~\cite{HerzigJZ13} and Kochhar et al.~\cite{KochharTL14} demonstrated that the proposed approach outperform the state-of-the-art bug localization solution based on deep learning recently proposed by Huo et al.~\cite{huo2016learning} and several other advanced baselines. \TRANPCNN\ can outperform the best performing baseline by ...\% to ...\% considering various standard evaluation metrics.

As a future work, we plan to extend the evaluation of \TRANPCNN\ by including more bug reports from additional projects (after a manual curation process similar to the ones performed by Herzig et al. and Kochhar et al.). We also plan to develop our solution into a tool that is integrated with an IDE followed by its evaluation by one of our industry partners. We also plan to further improve the performance of \TRANPCNN\ by considering data beyond text in bug reports and source code files. \dl{Xuan, please kindly help to complete the conclusion section too.}



\vspace{0.2cm}\noindent{\bf Acknowledgement and Replication Package.} Acknowledgement and link to replication package are omitted for double blind submission.

%\pagebreak
\balance
\bibliographystyle{ACM-Reference-Format}
\bibliography{ICSE18}

\end{document}
