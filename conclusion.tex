Recent supervised bug localization techniques make use of a training dataset of manually localized bugs to boost performance. Unfortunately, Zimmermann et al. have highlighted that sufficient defect data is often unavailable for many projects and companies~\cite{ZimmermannNGGM09}. Much defect data is also not clean and suffer from a variety of biases -- c.f.,~\cite{HerzigJZ13,KochharTL14,BachmannBRDB10,BirdBADBFD09}. To deal with this limitations, there is a need for a cross-project bug localization solution that can take data from a project to train a model for another project for which only limited clean bug data is available. To address this need, we propose a deep transfer learning approach specialized for bug localization named \TRANPCNN. \dl{Please add a sentence that remind readers about the novelty of the approach.} Experiments on manually curated datasets by Herzig et al.~\cite{HerzigJZ13} and Kochhar et al.~\cite{KochharTL14} demonstrated that the proposed approach outperform the state-of-the-art bug localization solution based on deep learning recently proposed by Huo et al.~\cite{huo2016learning} and several other advanced baselines. \TRANPCNN\ can outperform the best performing baseline by 61.13\% to 180.66\% considering various standard evaluation metrics.

As a future work, we plan to extend the evaluation of \TRANPCNN\ by including more bug reports from additional projects (after a manual curation process similar to the ones performed by Herzig et al. and Kochhar et al.). We also plan to develop our solution into a tool that is integrated with an IDE followed by its evaluation by one of our industry partners. We also plan to further improve the performance of \TRANPCNN\ by considering data beyond text in bug reports and source code files. \dl{Xuan, please kindly help to complete the conclusion section too.}

