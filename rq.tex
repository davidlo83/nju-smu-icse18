%\dl{Ferdian, please add details on why the RQs are interesting and the purpose of the RQs}

Our experiments are designed to address the following research questions:

\vspace{0.2cm}\noindent\textit{\textbf{RQ1}: Is there a need for a specialized technique for cross-project bug localization?}

If a model learned from one project can be used for other projects, then there is no need for a specialized technique for cross-project bug localization. Thus, before we consider other research questions, we validate the need for our proposed approach by empirically evaluating the effectiveness of a model learned from one project to localize bugs in other projects.

\vspace{0.2cm}\noindent\textit{\textbf{RQ2}: Do the heterogeneous predicting adaptation layers improve the bug localization performance?}

%\dl{Xuan, please help to motivate this research question. I can't motivate it since I believe the details of the approach is going to change.}

In Section 3, we propose to employ project-specific correlation fitting layers that apply two fully-connected networks for prediction from source and target projects separately, which is the key part of \TRANPCNN. In this research question, we evaluate whether the layers help improve the bug localization performance by comparing the results with NPCNN and SimpleTrans model. \ft{SimpleTrans not explained} 

\vspace{0.2cm}\noindent\textit{\textbf{RQ3}: Can \TRANPCNN outperform other bug localization methods?}

A number of bug localization methods have been proposed in the literature. In this research question, we evaluate whether and to what extent can our proposed approach \TRANPCNN outperforms the state-of-the-art methods designed for bug localization and those that can be adapted for bug localization. 